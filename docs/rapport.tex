\documentclass{article}
%\usepackage[utf8]{inputenc}
\usepackage[left=2.5cm,right=2.5cm,top=2.5cm,bottom=2.5cm]{geometry} % Page format
\usepackage{graphicx} % Pictures
\usepackage{float} % [H] parametre
\usepackage{fontspec} % Custom font
\usepackage[french]{babel} % French content
\usepackage{hyperref} % Clickable links
\usepackage{fourier-otf}
\usepackage{caption}
\usepackage{subcaption}
\usepackage{listings}
\usepackage[dvipsnames]{xcolor}
\usepackage{amsmath}
\usepackage{epigraph}
\input{commande}

%\setromanfont{Times New Roman}
\newcommand{\HRule}{\rule{\linewidth}{0.5mm}}
\setlength{\parskip}{1.8ex plus .9ex minus .4ex}
\setlength{\parindent}{0cm}

\begin{document}
\setlength{\parskip}{1.8ex plus .9ex minus .4ex}

\lstdefinestyle{mystyle}{
    backgroundcolor=\color[HTML]{f2f8fb},   
    commentstyle=\color{ForestGreen},
    keywordstyle=\color{blue},
    numberstyle=\tiny\color{gray},
    stringstyle=\color{purple},
    breaklines=true
    basicstyle=\ttfamily\footnotesize,
    breakatwhitespace=false,         
    breaklines=true,                 
    captionpos=b,                    
    keepspaces=true,                 
    numbers=left,                    
    numbersep=6pt,                  
    showspaces=false,                
    showstringspaces=false,
    showtabs=false,                  
    tabsize=2,
}

\lstset{style=mystyle}

\begin{titlepage}
    \centering
    \includegraphics[height = 3.7cm]{Image/logo_ISEN_CAMPUS_BREST.jpg}\\ [3cm]
    \textsc{\Large Gestion projet et qualité}\\ [0.5cm]
    \HRule \\ [0.4cm]
    {\Huge \textbf{Projet Cleanistère}}\\ [0.5cm]
    {\huge \textbf{SCRUM - Mise en application}}\\ [0.4cm]
    \HRule \\ 
    \large{Professeur : François Jaffrenou}\\ [2.5cm]
    {\large \textbf{Baptiste Fertille}}\\
    {\large \textbf{Alban Rouillé}}\\
    {\large \textbf{Jonathan Sling}}\\
    {\large \textbf{Augustin De Toffoli}}\\[2cm]
    {\large \textbf{\today}}\\
    {CIPA 4 Brest - Année académique 2024-2025}\\ [0.5cm]
    \hspace{-2.5cm}
    \includegraphics[height = 3cm]{Image/Yncrea_ouest.png}
    \hspace{5cm}
    \includegraphics[height = 3cm]{Image/cti.png}\\
\end{titlepage}

\newpage
\mbox{}

\newpage
\tableofcontents

\newpage
\mbox{}

\newpage


\section{Le projet Cleanistère}
\subsection{Description}
Ce projet vise à structurer et développer les actions de ramassage des déchets sur l’ensemble du littoral de la rade de Brest, de Plouzané à Plougastel-Daoulas. Porté par une association environnementale locale, il s’appuie sur un outil numérique participatif destiné à fédérer une communauté de bénévoles, faciliter l’organisation et la visibilité des initiatives, et renforcer l’engagement citoyen en faveur de la protection du littoral. Intuitif et ouvert à tous, cet outil se veut un levier concret d’impact environnemental positif.
\subsection{Vision}
\begin{quotation}
    \texttt{Faire du Finistère un modèle de littoral préservé et sans déchets en transformant chaque citoyen en acteur du changement grâce à une plateforme numérique fédératrice et intuitive}
\end{quotation}
\subsection{Logo et charte graphique}
\begin{figure}[H]
    \centering
    \begin{subfigure}{0.49\textwidth}
        \includegraphics[width=\textwidth]{logo_Cleanistère.png}
        \caption{logo}
        \label{logo_cleanistere}
    \end{subfigure}
    \hfill
    \begin{subfigure}{0.49\textwidth}
        \fbox{\includegraphics[width=\textwidth]{palette.png}}
        \vspace{2.5cm}
        \caption{Palette light and dark mode}
        \label{palette_cleanistère}
    \end{subfigure}
            
    \caption{Charte graphique de Cleanistère}
    \label{colors_palette}
\end{figure}

\subsection{Persona et user story} 
- Marie, la citoyenne engagée\\
- Julien, le professeur\\
- Claire, la maman écoresponsable\\
- Luc, le bénévole expérimenté / militant\\
- Sophie, conseillère municipale\\

Une description plus détaillée peut-être trouvée en annexe \ref{Persona}


\section{Organisation interne}
\begin{itemize}
    \item Product Owner : Augustin De Toffoli
    \item Scrum Master : Jonathan Sling
    \item Developpement Team : Alban Rouillé
    \item Developpement Team : Baptiste Fertille
\end{itemize}


\section{Organisation du projet}
\subsection{Methodologie}\label{organisation}
Le première chose à faire est de décider, sur base des users story de nos persona (cf. \ref{Persona}), les fonctionnalités à implémenter. Pour ne pas se perdre, chaque fonctionnalité est labelisé du persona et de l'user story d'où elle vient.
L'accès au backlog peut se faire via ce lien : \href{https://github.com/users/Ramdek/projects/1}{\underline{\textit{{Backlog GitHub}}}}

Une fois les backlog fixés, les issues sont classées par priorité pour baliser le projet. Nous avons choisis de répartir notre fichier en 3 niveaux de priorité. 
Les Milestones du projet sont disponible via ce lien : \href{https://github.com/Ramdek/cleanistere/milestones?sort=title&direction=asc}{\underline{\textit{{Milestones GitHub}}}}

Avant de commencer les sprints, la complexité de chaque issue est estimé en niveau (XS, S, M, L, XL)

\subsubsection{Definition of ready (DOR)}
La DOR représente les critères qu'une User Story devra impérativement remplir avant d'être incluse dans un sprint lors du Sprint Planning.\\
Nous avons retenu les critères suivants :
\begin{itemize}
    \item Valeur métier claire : La priorité et l'objectif pour l'utilisateur sont bien définis.
    \item Description complète : L'US respecte le format « En tant que... Je souhaite... Car... » et est dénuée d'ambiguïté.
    \item Critères d'acceptation définis : Les conditions de succès sont explicitement listées.
    \item Estimée : La complexité a été évaluée par l'équipe.
    \item Petite taille : L'histoire est suffisamment découpée pour être réalisable en un seul sprint.
    \item Dépendances identifiées : Aucun obstacle externe majeur ne bloque le démarrage de la tâche.
\end{itemize}

 

\subsubsection{Definition of Done (DOD)}
La DOD est la liste utilisée pour déclarer qu'une fonctionnalité est totalement terminée et constitue un incrément potentiellement livrable.

\begin{itemize}
    \item Critères d'acceptation validés : Tous les tests fonctionnels définis initialement sont passés avec succès.
    \item Qualité technique : Le code est intégré, revu et ne présente pas de défauts majeurs.
    \item Revue UX terminée : L'interface respecte la charte graphique et la maquettes validée.
    \item Documentation à jour : Les informations nécessaires au projet ont été consignées dans le rapport ou les annexes.
    \item Validation finale : Le PO a inspecté l'incrément lors de la review.
\end{itemize}


\newpage
\subsection{Sprints}
Le projet sera réalisé en 4 sprint de chacun environ 1.5h
\subsubsection{Sprint 1 : Fondations et Identité Visuelle}
\ssssection{Objectif}
L'objectif de ce sprint était d'établir l'architecture technique et l'identité visuelle du projet

\ssssection{Déroulement}
Lors du Sprint Planning, le Product Owner a clarifié la vision. La Dev Team s'est concentrée sur la création de la maquette Excalidraw et la définition de la structure de données.\\
Vous pouvez retrouver la maquette du site en annexe \ref{maquette}

\begin{itemize}
    \item Création de la maquette
    \item Creation de la charte graphique du projet (cf. figure \ref{colors_palette})
    \item Spécification détaillées des issues (cf. \ref{organisation})
    \item Structure des données
    \item Répartissions des taches
\end{itemize}

\ssssection{Retrospective}
La première retrospective a permis de valider la charte graphique (palette "Littoral" et logo). La Rétrospective a servi à ajuster la communication entre les développeurs pour éviter les doublons dans le code.

\subsubsection{Sprint 2 : Mise en œuvre du MVP (Minimum Viable Product)}
\ssssection{Objectif}
Développer la colonne vertébrale du site, comprenant la base de donnée, la structure du site et les bases de la cartes interactive.

\ssssection{Déroulement}
Lors du planning, il a été décidé de se concentrer, vu le peu de temps disponible, de se concentrer sur les bases de viabilité fonctionnel du site.
\begin{itemize}
    \item Développement web de la maquette
    \item Intégration de la  carte interactive 
    \item Réalisation de la base de donnée
    \item Mise en ligne du site
\end{itemize}

\ssssection{Retrospective}
La retrospective a mit en évidence les fonctionalités manquantes de la carte interactive et permis la mise au point de la communication avec la base de donnée.

\begin{figure}[H]
    \centering
    \includegraphics[width=1\linewidth]{instant_project.png}
    \caption{instantané de l'évolution du projet à la fin du sprint 2}
    \label{instant_project_evol}
\end{figure}

\subsubsection{Sprint 3: Consolidation des bases}
\ssssection{Objectif}
Transformer l'application en un outil de gestion personnalisé pour pérenniser l'engagement des bénévoles et des partenaires.

\ssssection{Déroulement}
\begin{itemize}
    \item Systeme de login mis en place
    \item Amélioration de la cartographie
    \item Ajout de statistiques
    \item Page de présentation du projet
\end{itemize}

\ssssection{Retrospective}
L'équipe s'est rendu compte que les statistiques et le systeme de login prennent plus longtemps que prévu à être réalisé.

\subsubsection{Sprint 4 : Finalisation}
\ssssection{Objectif}
Finaliser la page pour lui donner une robustesse vis à vis des bugs qui pourraient survenir

\ssssection{Déroulement}
Le dernier sprint a été dédié à la recherche de bugs ou d'erreur qui pourraient survenir et à les résoudre.

\begin{itemize}
    \item Finalisation des statistiques
    \item Finalisation de la connexion
    \item Ajout de la fonctionalité d'ajout d'évenement
    \item Debug des fonctionalités
\end{itemize}

\ssssection{Retrospective}
Le projet a révelé des bug inatendut présentant des difficultés à être résolue

\begin{figure}
    \centering
    \includegraphics[width=1\linewidth]{instantané_final.png}
    \caption{Instantané final}
\end{figure}

\subsection{Démarche technique}
\subsubsection{HTML et Javascript}
Nécessaire pour la construction d'un site
\subsubsection{CSS}
Utilisation de TailWindCSS pour accélérer le développement des visuels du site



\newpage
\section{Annexes}
\subsection{Persona}\label{Persona}
Les personas et User story peuvent-être trouvées sur \href{https://github.com/Ramdek/cleanistere/blob/main/docs/persona.md}{\underline{\textit{{Persona et user story GitHub}}}}
\begin{figure}[H]
    \centering
    \vspace{-2cm}
    \includegraphics[width=0.85\linewidth]{persona_visu.png}
    \caption{Visuel de 4 persona}
    \label{Visu_persona}
\end{figure}

\subsection{Maquette}\label{maquette}
\begin{figure}[H]
    \centering
    \includegraphics[width=.9\linewidth]{maquette.png}
    \caption{Maquette du site}
\end{figure}

\end{document}


